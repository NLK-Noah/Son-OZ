\documentclass[a4paper,11pt]{article}
\usepackage[utf8]{inputenc}
\usepackage[T1]{fontenc}
\usepackage[french]{babel}
\usepackage{geometry}
\usepackage{hyperref}
\usepackage{enumitem}
\geometry{margin=2.5cm}

\title{\textbf{LSINC1104 – Son’OZ Project 2025}\\Rapport de Projet}
\author{
  Kaan Akman\\
  NOMA: 0910-23-00
  \and
  Noah Moussaoui\\
  NOMA: 8231-23-00
}
\date{19 avril 2025}

\begin{document}

\maketitle

\section*{1. Limitations et problèmes connus}

Nous n’avons pas rencontré de problèmes majeurs qui pourraient nuire au fonctionnement global du projet.

Cependant, nous n’avons pas pu implémenter l’extension \texttt{instrument}. De plus, lorsque nous lançons la commande \texttt{make run} avec une composition différente de \texttt{joy.dj.oz}, un message d’erreur peut apparaître dans \texttt{main.oz}, ce qui reste assez mystérieux. Heureusement, cela n’empêche pas la génération du fichier \texttt{out.wav}, et la mélodie reste bien jouable.

Nous avons également été confrontés à quelques confusions au niveau des signatures des filtres (comme \texttt{fade}, \texttt{cut}, etc.), notamment sur l’usage ou non de labels dans les arguments. Cela a entraîné quelques essais-erreurs lors de l’écriture des tests.

\section*{2. Constructions non-déclaratives}

Nous n’utilisons aucune construction non-déclarative dans notre projet : pas de cellule, port ou thread. Le projet repose exclusivement sur des constructions déclaratives, avec du pattern matching, des définitions locales via \texttt{local...in...end}, et des appels récursifs.

\section*{3. Choix d’implémentation surprenants}

Nous avons veillé à écrire un code compréhensible pour tout lecteur, avec des noms de variables explicites et des commentaires clairs.

Nous n'avons pas pris de décisions d’implémentation particulièrement surprenantes. Nous avons simplement respecté la structure recommandée dans l’énoncé, en nous appuyant autant que possible sur la documentation officielle de Mozart/Oz.

\section*{4. Extensions réalisées}

\subsection*{Reverse}

L’unique extension que nous avons implémentée est le filtre \texttt{reverse}. Celui-ci permet d’inverser une musique, en renversant l’ordre des échantillons. Cela peut servir à produire des effets sonores spéciaux ou des transitions créatives.

L’extension est activée via une variable booléenne \texttt{ActivatorOfExtensions}, définie dans le fichier \texttt{Mix.oz}. Lorsqu’elle vaut \texttt{true}, le filtre \texttt{reverse} est appliqué. Si elle vaut \texttt{false}, il est ignoré (afin de ne pas interférer avec les tests automatiques).

Malheureusement, nous n'avons pas réussi à rendre ce commutateur dynamique via le \texttt{Makefile}. Nous avons également rencontré des difficultés techniques pour partager cette variable entre fichiers, comme entre \texttt{Mix.oz} et \texttt{Tests.oz}. En conséquence, la variable est dupliquée dans les deux fichiers, et son état doit être modifié manuellement.

L’extension est testée dans \texttt{Tests.oz} sur un exemple simple avec trois échantillons.

\section*{Conclusion}

Nous avons respecté les consignes de l’énoncé et les signatures de code données. Malgré quelques difficultés liées à la compréhension du langage Oz, des formats de structures et des messages du compilateur, nous avons pu livrer un projet complet et fonctionnel.

Le projet a été entièrement réalisé par nous-mêmes, sans recours à des générateurs de code externes. Nous avons appris à structurer du code déclaratif, gérer des partitions musicales complexes, et concevoir un moteur audio miniature en Oz.

\end{document}